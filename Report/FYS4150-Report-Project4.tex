\documentclass{article}

\usepackage[margin=0.5in,bottom=1in,footnotesep=1in]{geometry}

\usepackage{amsmath}


\usepackage{multicol}
\setlength{\columnsep}{1cm}
\usepackage[]{algorithm2e}

\usepackage{lipsum}% for dummy text
\usepackage[varg]{txfonts}
\usepackage{graphicx}
\usepackage{subcaption}
\usepackage{multirow}

\usepackage[font=small,labelfont={sf,bf}]{caption}

\usepackage{color}

\usepackage[export]{adjustbox}

\usepackage{titlesec}
\titleformat{\section}{\fontfamily{phv}\fontsize{12}{15}\bfseries}{\thesection}{1em}{}
\titleformat{\subsection}{\fontfamily{phv}\fontsize{10}{15}\itshape}{\thesubsection}{1em}{}
\titleformat{\subsubsection}{\fontfamily{phv}\fontsize{9}{15}\bfseries}{\thesubsubsection}{1em}{}


\title{\textbf{FYS4150 Project 4: \\The Ising model in two dimensions}}
\author{Marie Foss (\# 56), Maria Hammerstr{{\o}}m (\# 59)}
\date{}

\begin{document}

\maketitle

\begin{abstract}
	\noindent We study the Ising model for a spin system in two dimensions for different lattice sizes $L$ and estimate the critical temperature $T_{\mathrm{C}}$ of the system as it experiences a magnetic phase transition using Monte Carlo simulations. A lattice size of $L = 80$ allows us to estimate a critical temperature with a relative error of $1.36 \times 10^{-2}$ compared to the exact value of 2.269 (in units of $kT/J$). We find that using a value of $L = 80$ is sufficient to get good results without the code taking too long to compile, using a temperature step length of 0.05. Whether we initialize the system with a random or ordered spin matrix is of no importance after about 1000 Monte Carlo cycles at the highest, when all the expectation values of interest has reached the most likely state. Here we study the behavior of the expectation values for the energy, magnetization, susceptibility and heat capacity. We find that these expectation values oscillate more for higher temperatures as a function of the number of Monte Carlo cycles (analogous with the passing of time), resulting in a large value for the variance $\sigma_E^2$ and a small value for the probability $P(E)$ for getting an energy equal to the expectation value for the energy, as opposed to temperatures well below the critical value where the value of $\sigma_E^2$ will be low and $P(E)$ will be high.
	% \lipsum[1]
	\vspace*{2ex}
	
	\noindent \textbf{Github:} \textit{https://github.com/mariahammerstrom/Project4}
	\vspace*{2ex}
\end{abstract}



\begin{multicols}{2}

\section{Introduction}

In this project we study the Ising model in two dimensions without an external magnetic field. The Ising model is used to study phase transitions at finite temperature for magnetic systems. The model was invented in 1920\footnote{Lenz, W. (1920). "Beitr�ge zum Verst�ndnis der magnetischen Eigenschaften in festen K�rpern". \textit{Physikalische Zeitschrift} 21: 613-615.} and the analytical solution to the two-dimensional case was found in 1944\footnote{Onsager, Lars (1944). "Crystal statistics. I. A two-dimensional model with an order-disorder transition". \textit{Physical Review}, Series II 65 (3-4): 117-149.}.




\subsection{Theory}\label{subsec:theory}
In the two-dimensional Ising model, the simplest form of the \textbf{energy} for a specific microstate $i$ is expressed as

\begin{equation}\label{eq:energy}
	E_i = -J \sum_{\langle kl\rangle}^{N}s_k s_l,
\end{equation}
with  $s_k=\pm 1$ where $+ 1$ denotes spin up and $ - 1$ denotes spin down, $N$ is the total number of spins and $J$ is a coupling constant expressing the strength of the interaction between neighbouring spins. The symbol $\langle kl\rangle$ indicates that the sum is over nearest neighbours only. We assume ferromagnetic ordering, meaning $J> 0$ and will make use of periodic boundary conditions.

The probability of finding the system in a given microstate $i$ is expressed by the \textbf{Boltzmann probability distribution}

\begin{equation}
	P_i (\beta) = \frac{e^{-\beta E_i}}{Z},
\end{equation}
where $\beta = 1/kT$ with $k$ the Boltzmann constant and $T$ the temperature, and with $Z$ as the \textbf{partition function} for the canonical ensemble defined as

\begin{equation}\label{eq:partition_func}
	Z = \sum_{i = 1}^{\infty} e^{-\beta E_i},
\end{equation}
summing over all micro states $i$. 

The \textbf{magnetic moment} of a given microstate is

\begin{equation}\label{eq:magnetization}
	{\cal M}_i = \sum_j s_j.
\end{equation} 
Some quantities of interest are the \textbf{expectation values} for the energy $\langle E\rangle$ and magnetic moment $\langle {\cal M}\rangle$:

\begin{equation}
\begin{aligned}
	\langle E \rangle &= \frac{1}{Z} \sum_i E_i e^{- \beta E_i} = kT^2 \frac{\partial \textrm{ ln } Z}{\partial T} = - \frac{\partial \textrm{ ln } Z}{\partial \beta} \\
	\langle {\cal M}\rangle &= \frac{1}{Z} \sum_i {\cal M}_i e^{- \beta E_i},
\end{aligned}
\end{equation}
as well as the \textbf{variances} for the energy $\sigma_E^2$ and for the magnetic moment $\sigma_M^2$, describing how the calculated values of $E$ and $M$ deviates from the expectation values:

\begin{equation}\label{eq:expect_values}
\begin{aligned}
	\sigma_E^2 &= \langle E^2\rangle - \langle E\rangle^2 \\
	\sigma_{\cal M}^2 &= \langle {\cal M}^2\rangle - \langle {\cal M}\rangle^2,
\end{aligned}
\end{equation}
These values can be used to calculate the \textbf{heat capacity} of a fixed volume given by

\begin{equation}
	C_V = \frac{\sigma_E^2}{kT^2} = \frac{\partial \langle E\rangle}{\partial T},
\end{equation}
and the \textbf{susceptibility}, which describes whether a material is attracted into or repelled out of a magnetic field, given by

\begin{equation}
	\chi = \frac{\sigma_{\cal M}^2}{kT} = \frac{\langle {\cal M}^2\rangle - \langle {\cal M}\rangle^2}{kT}.
\end{equation}
We want to compute these quantities after the system has \textbf{thermalized}, which is when the system has reached its most likely state. This will depend on the temperature $T$.

The model we are considering here undergoes a \textbf{phase transition}. Below a critical temperature $T_{\mathrm{C}}$ there is spontaneous magnetization $\langle {\cal M}\rangle \neq 0$ (magnetic phase), while above this temperature the average magnetization is zero (paramagnetic phase). Near $T_{\mathrm{C}}$ we can characterize the behaviour of many physical quantities by a power law behaviour.

An important quantity is the \textbf{correlation length}, which is expected to be of the order of the lattice spacing for $T \gg T_{\mathrm{C}}$. Because the spins become more and more correlated as $T$ approaches $T_{\mathrm{C}}$, the correlation length increases as we get closer to the critical temperature. The divergent behaviour of $\xi$ near $T_{\mathrm{C}}$ is

\begin{equation}\label{eq:xi}
	\xi(T) \sim \left|T_{\mathrm{C}} - T\right|^{-\nu}.
\end{equation}
A second-order phase transition is characterized by a correlation length which spans the whole system. Since we are always limited to a finite lattice, $\xi$ will be proportional with the size of the lattice. Through so-called finite size scaling relations it is possible to relate the behaviour at finite lattices with the results for an infinitely large lattice. The critical temperature then scales as

\begin{equation}\label{eq:T_C}
	T_{\mathrm{C}}(L) - T_{\mathrm{C}}(L=\infty) = aL^{-1/\nu},
\end{equation}
with $a$ a constant and $\nu$ defined in Eq.~(\ref{eq:xi}). In this project we will use $\nu = 1$.




\subsection{Analytical solution}\label{subsec:analytical}
First we will assume that we only have two spins in each dimension, that is $L = 2$, where $L$ is the width of the lattice. The situation looks like this:

\begin{align*}
	\uparrow_{(1)} \quad &\uparrow_{(2)}   \\
	\uparrow_{(3)} \quad &\uparrow_{(4)}\,,   
\end{align*}
where an upward arrow denotes spin up and the numbers are just a way to identify the different spins. We will make use of \textbf{periodic boundary conditions}, which means that the neighbour to the right of a given spin $s_N$ takes the value of $s_1$. Similarly, the neighbour to the left of $s_1$ takes the value of $s_N$. This way of treating the boundaries are often used when approximating an infinite system that has a repeating structure. In our case, this means that we will treat our system as though it looked like this (where the original system is highlighted in magenta):

\begin{align*}
	\textcolor{black}{\uparrow_{(4)}} \quad \textcolor{black}{\uparrow_{(3)}} \quad \textcolor{black}{\uparrow_{(4)}} \quad &\textcolor{black}{\uparrow_{(3)}} \\
	\textcolor{black}{\uparrow_{(2)}} \quad \textcolor{magenta}{\uparrow_{(1)}} \quad \textcolor{magenta}{\uparrow_{(2)}} \quad &\textcolor{black}{\uparrow_{(1)}} \\
	\textcolor{black}{\uparrow_{(4)}} \quad \textcolor{magenta}{\uparrow_{(3)}} \quad \textcolor{magenta}{\uparrow_{(4)}} \quad &\textcolor{black}{\uparrow_{(3)}} \\
	\textcolor{black}{\uparrow_{(2)}} \quad \textcolor{black}{\uparrow_{(1)}} \quad \textcolor{black}{\uparrow_{(2)}} \quad &\textcolor{black}{\uparrow_{(1)}}\,.
\end{align*}
Closed form expression can be found for the partition function in Eq. (\ref{eq:partition_func}), and the corresponding expectation values for $E$, $|{\cal M}|$, the specific heat $C_V$ and the susceptibility $\chi$ as functions of $T$ using periodic boundary conditions. In the case $L = 2$ we can write Eq. (\ref{eq:energy}) for the different components, making use of the periodic boundary conditions, giving

\begin{align*}
	E_1 &= - J (s_1 s_2 + s_1 s_3) = - 2J, \\
	E_2 &= - J (s_1 s_2 + s_2 s_4) = - 2J, \\
	E_3 &= - J (s_1 s_3 + s_3 s_4) = - 2J, \\
	E_4 &= - J (s_3 s_4 + s_2 s_4) = - 2J. 
\end{align*}
Thus the total energy for the system is

\begin{equation*}
	E = E_1 + E_2 + E_3 + E_4 = - 8J.
\end{equation*}
If going through this exercise for different configurations of spin up and spin down, we find values for energies, degeneracies and magnetization for different configurations as shown in Table~\ref{table:quantities}, where magnetization is calculated from Eq. (\ref{eq:magnetization}).

\begin{table*}
\begin{center}
\begin{tabular}{ l r r r }\hline
	Number of spins up 			& Energy $E$	 			& Degeneracy $\Omega$		& Magnetization ${\cal M}$		\\ \hline
	4 						& $- 8J$ 					& 1						& 4		 \\
	3 						& 0						& 4						& 2		 \\
	2						& 0						& 4						& 0		\\
	2						& $8J$					& 2						& 0		\\
	1						& 0						& 4						& -2		\\
	0						& $-8J$					& 1						& -4		\\
	\hline
\end{tabular}
\caption{Energy and magnetization for the two-dimensional Ising model with $N = 2 \times 2$ spins with periodic boundary conditions.}\label{table:quantities}
\end{center}
\end{table*}

These calculations can be used to find the closed-form expressions, starting with the partition function, which can be expressed as

\begin{align}\label{eq:Z_analytic}
	Z(\beta) 	&= \sum_E \Omega(E) e^{- \beta E} = 2 e^{8J\beta} + 2 e^{- 8J \beta} + 12 \\
			&= 4 \textrm{ cosh}(8J \beta) + 12.
\end{align}
Then the expectation value for the energy can be written, using the expression in Eq. (\ref{eq:expect_values}), as

\begin{equation}\label{eq:E_avg_analytic}
	\langle E\rangle = - \frac{8J \textrm{ sinh}(8J \beta)}{\textrm{cosh}(8J\beta) + 3} \approx - 8J \textrm{ tanh} (8J\beta),
\end{equation}
where we have used that cosh$(8J\beta) \gg 3$ in the approximation. Using this approximation gives

\begin{equation}\label{eq:C_V_analytic}
	C_V(\beta) = k \textrm{ } \bigg( \frac{8J\beta}{\textrm{cosh }(8J\beta)} \bigg)^2.
\end{equation}
Similarly, using the values for magnetization in Table~\ref{table:quantities} and Eq. (\ref{eq:magnetization}), gives

\begin{align}\label{eq:M_avg_analytic}
	\langle|{\cal M}|\rangle 	&= \frac{1}{Z} \bigg( 4e^{8J\beta} + 4\cdot2 + 4\cdot|-2| + |-4|e^{8J\beta} \bigg) \\
						&= \frac{8}{Z}(e^{8J\beta} + 2),
\end{align}
and
\begin{align}
	\langle {\cal M}\rangle 	&= \frac{1}{Z} \bigg( 4e^{8J\beta} + 4\cdot2 - 4\cdot2 - 4e^{8J\beta} \bigg) = 0, \\
	\langle {\cal M}^2\rangle 	&= \frac{1}{Z} \bigg( 4^2 e^{8J\beta} + 4\cdot2^2 + 4\cdot(-2)^2 + (-4)^2e^{8J\beta}   \bigg) \\
						&= \frac{32}{Z} \bigg( e^{8J\beta} + 1  \bigg).
\end{align}
Thus
\begin{equation}\label{eq:chi_analytic}
	\chi(\beta) = \frac{\langle {\cal M}^2\rangle - \langle {\cal M}\rangle^2}{kT} = \frac{32\beta}{Z} [e^{8J\beta} + 1] - \frac{\beta}{Z^2}[8e^{8J\beta} + 4]^2.
\end{equation}









\section{Methods}

We wrote a code for the Ising model in two dimensions which computes the mean energy $E$, mean magnetization $|{\cal M}|$, the specific heat $C_V$ and the susceptibility $\chi$ as functions of  $T$ using periodic boundary conditions for a lattice of size $L$ in the $x$ and $y$ directions. The total number of configurations is given by $2^N$ with $N = L \times L$ the number of spins for a lattice of width $L$. 

\subsection{Algorithm}\label{subsec:metropolis}
We made use of the \textbf{Metropolis algorithm}:

\begin{enumerate}
	\item Generate a random configuration in the lattice to create an initial state with energy $E_b$ ($b$ = beginning state).
	\item Change the initial configuration by flipping one spin. Compute the energy of this state $E_t$ ($t$ = trial state).
	\item Calculate $\Delta E = E_t - E_b$.
	\item \textit{The Metropolis test:} If $\Delta E \leq 0$ the new configuration is accepted, meaning the energy is lowered and we are moving towards the energy minimum at a given temperature. If $\Delta E > 0$, calculate $w = e^{- \beta \Delta E}$. If $w < r$ where $r$ is a random number between 0 and 1, accept the new configuration. Keep the old configuration otherwise.
	\item Compute the new energy $E' = E_t + \Delta E$. Calculate various expectation values using this energy.
	\item Repeat steps $2-5$ for the chosen number of Monte Carlo (MC) cycles, meaning the number of times we should sweep through the lattice.
	\item Compute the various expectation values by dividing by the total number of cycles (and possibly the number of spins).
\end{enumerate}
During the computation we want to check how many MC cycles we need before the system has reached its most likely state. This is done by comparing the new value of $E$ with the previous value of $E$. If the difference is smaller than $1 \%$, we can say that the most likely state has been reached. (We had to choose $1 \%$ rather than the suggested $5 \%$ because the first two energies in the $T = 2.4$ case vary less than $5 \%$.)

We run 1 million MC cycles in total.


\subsection{Parallelization}
We have parallelized our code using the Open MPI library. 


\subsection{Calculating $T_{\mathrm{C}}$}\label{subsec:T_C}
The critical temperature $T_{\mathrm{C}}$ is described by Eq. (\ref{eq:T_C}), where $T_{\mathrm{C}}(L)$ is calculated for different lattice sizes $L$ through the Metropolis algorithm described above.  $T_{\mathrm{C}}(L = \infty)$ is described by the exact value of 2.269 (in units of $kT/J$). Eq. (\ref{eq:T_C}) has two unknowns, $T_{\mathrm{C}}(L = \infty)$ and the constant $a$. By using our calculations for two different lattice sizes to calculate $T_{\mathrm{C}}(L)$, we can solve Eq. (\ref{eq:T_C}) in the two cases and get an estimate for $a$ and thus the critical temperature $T_{\mathrm{C}}$, using

\begin{equation}\label{eg:T_C_estimate}
	T_{\mathrm{C}} = T_{\mathrm{C}}(L) - \frac{a}{L}.
\end{equation}



\section{Results \& comments}\label{sec:results}

\subsection{Checking the 2 $\times$ 2 case}

Computing the mean energy $E$, mean magnetization $|{\cal M}|$, the specific heat $C_V$ and the susceptibility $\chi$ numerically with periodic boundary conditions, using the expressions from Sec.~\ref{subsec:theory}, gives results which can be compared with the results of the analytical closed-form expressions given in Sec.~\ref{subsec:analytical} to check that our code is working properly. 

For temperature $T=1.0$ (in units of $kT/J$), this gives the following results for a $2 \times 2$ spin system where the initial state has all spins pointing upwards:

\begin{center}
\begin{tabular}{ l l l }\hline
	Quantity 								& Closed form	 				& Numerical		\\ \hline
	Mean energy $E$ 						& $-8.00$ 						& $-8.00$		 \\
	Mean magnetization $|{\cal M}|$ 			& 3.99						& 4.00		 \\
	Specific heat $C_V$						& $2.88 \times 10^{-5}$			& 0.00 		\\
	Susceptibility $\chi$						& $3.21 \times 10^{-2}$			& 0.00		\\
	\hline
\end{tabular}
\end{center}
These results follow closely that of Table \ref{table:quantities}. 

The total number of MC cycles needed to achieve good agreement for the expectation value of the energy is only one cycle, which gives perfect results according to Table~\ref{table:quantities}. We see that the results from the closed-form expressions are slightly off, which is probably due to the approximation we used in Eq. (\ref{eq:E_avg_analytic}). 


\begin{figure*}[t]
\begin{center}
\begin{tabular}{ccc}
  	\includegraphics[width=85mm]{images/E_avg_MC_L20.png}
	& \includegraphics[width=85mm]{images/M_avg_MC_L20.png} \\
	(a) Energy					& (b) Magnetization  \\
	
	 \includegraphics[width=85mm]{images/chi_MC_L20.png}
	& \includegraphics[width=85mm]{images/Cv_MC_L20.png} \\
	(c) Susceptibility					& (d) Heat capacity  \\[6pt]
\end{tabular}
\caption{Expectation values as a function of number of MC cycles with the hottest temperature in red (for an ordered initial spin state)).}\label{fig:L20_ex_values}
\end{center}
\end{figure*}


\subsection{The 20 $\times$ 20 case}
Now that we have confirmed that our code produces results comparable to the analytic solution, we can look at the case of $L = 20$ spins in the $x$ and $y$ directions. In this case we wanted to perform a closer study of the time (or number of MC cycles) we need before reaching an equilibrium situation and can start computing the various expectation values. A rough study is done by plotting the various expectation values as a function of number of MC cycles, for both $T = 1.0$ and $T = 2.4$, as shown in Fig.~\ref{fig:L20_ex_values}a-d. 

For $T = 1.0$, we see that the expectation values does not change considerably with the number of MC cycles, meaning the system reached the equilibrium state quickly. This is because we set all spin configurations to spin up for temperatures lower than $T=1.5$, meaning the system will be in its ground state of minimal energy right away. Here the minimum number of MC cycles is around 2, according to the method described in Sec.~\ref{subsec:metropolis}. This number will vary slightly for each run due to the use of random numbers in the method.

For $T = 2.4$, on the other hand, the expectation values shows a larger variation for small numbers of MC cycles in Fig.~\ref{fig:L20_ex_values}a, meaning the system reaches an equilibrium at a later point. In this case the minimum number of MC cycles is around 11 when using the method in Sec.~\ref{subsec:metropolis} where we compare the new energy value with the previous one. When studying the graphs in Fig.~\ref{fig:L20_ex_values}a we see that the minimum number of MC cycles would be much larger if comparing energy values from a larger intervall than one MC cycle. Then the minimum number of MC cycles would be around 2000 MC cycles according to the figure.

Do these conclusions depend on whether we have an ordered (all spins pointing in one direction) or a random spin matrix as the starting configuration? Fig.~\ref{fig:random}a-b shows that for the low temperature of $T = 1.0$ the randomly oriented spin matrix will take longer to reach the most likely state, while the ordered spin matrix will hit the expectation value right away. In the random case, we need about 1000 MC cycles before the most likely state is reached, according to the figure.

In the case of $T = 2.4$ both configurations deviate from the long-term expectation value in the beginning, then slowly oscillates towards a value close to the expectation value. This, again, happens after about 1000 MC cycles. 


\begin{figure*}[t]
\begin{center}
\begin{tabular}{c c c}
  	\includegraphics[width=85mm]{images/E_avg_MC_L20_random_T10.png}
	& \includegraphics[width=85mm]{images/M_avg_MC_L20_random_T10.png} \\
	(a) Energy	, $T = 1.0$				& (b) Magnetization, $T = 1.0$  \\
	
	 \includegraphics[width=85mm]{images/E_avg_MC_L20_random_T24.png}
	& \includegraphics[width=85mm]{images/M_avg_MC_L20_random_T24.png} \\
	(c) Energy, $T = 2.4$				& (d) Magnetization, $T = 2.4$  \\[6pt]
\end{tabular}
\caption{Expectation values as a function of \# of MC cycles with random and non-random initial spin matrix.}\label{fig:random}
\end{center}
\end{figure*}



\subsection{The number of accepted configurations}

The number of accepted configurations as a function of the total number of MC cycles is shown in Fig.~\ref{fig:L20_accepted_configs}a. The plot shows the number of accepted configurations divided by the total number of MC cycles. The number of accepted configurations scales approximately linearly with the total number of MC cycles. For $T=1.0$ there are no accepted configurations since no spin in the lattice is changed from up to down. 

For the case of $T = 2.4$ the number of accepted configurations increases rapidly in the beginning, then stabilizing, as the most likely state has been reached. 

\begin{figure*}[t]
\begin{center}
\begin{tabular}{ccc}
  	\includegraphics[width=85mm]{images/accepted_configs_MC_L20.png}
	& \includegraphics[width=85mm]{images/accepted_configs_temp.png} \\
	(a) As a function of \# of MC cycles.					& (b) As a function of temperature.  \\[6pt]
\end{tabular}
\caption{Number of accepted configurations.}\label{fig:L20_accepted_configs}
\end{center}
\end{figure*}
The total number of accepted configurations increases with higher temperature as shown in Fig.~\ref{fig:L20_accepted_configs}b. This is because of the test in the Metropolis algorithm saying that a configuration is accepted if $r < e^{-\Delta E/kT}$ where $r$ is a random number between 0 and 1. When the temperature is higher, $e^{-\Delta E/kT}$ will have a higher value, thus the probability of a spin flip is larger (as many more values of $r$ will fulfill the test), leading to more accepted configurations.

The figure also shows that this increase is greater with higher lattice size. This is as expected, as a larger lattice size means that we have a larger number of spins that can change direction, creating the possibility of many more accepted configurations. 



\subsection{The probability P(E)}

\begin{figure*}[b!]
\begin{center}
\begin{tabular}{c c}
  	\includegraphics[width=85mm]{images/prob_dist_T10.png}
	& \includegraphics[width=85mm]{images/prop_dist_T24.png} \\
	(a) $T = 1.0$					& (b) $T = 2.4$  \\ [6pt]
\end{tabular}
\caption{The probability distribution $P(E)$ for the energies $E$ for different temperatures.}\label{fig:prob_dist}
\end{center}
\end{figure*}

By simply counting the number of times a given energy appears in our computation, we can compute the probability distribution $P(E)$. We start the counting after the steady state situation has been reached, still looking at the $L = 20$ case. Fig.~\ref{fig:prob_dist}a-b shows the probability distribution for temperatures $T = 1.0$ and $T = 2.4$. 

We see that for the low temperature the probability of getting an energy corresponding to the value of the expectation value for the energy, $\langle E\rangle$, is very high, and the spreading of energies is small, which corresponds with the small value of the variance $\sigma_E^2 = 0.02$.

The opposite is true in the case of the higher temperature, shown in Fig.~\ref{fig:prob_dist}b. The probability is still highest for $\langle E\rangle$, but the spreading of the energies around this value is much larger, corresponding with the variance of $\sigma_E^2 = 8.28$. 

In conclusion, this means that where there is a small deviation in energy values, described by the variance, there will be a high probability of getting the expectation value $\langle E\rangle$. When the deviation is large as for $T=2.4$, the probability of getting the value $\langle E\rangle$ is small.

The numbers presented here apply in the case of an ordered initial spin matrix, but as follows from Fig.~\ref{fig:random} this result is not much different for a randomly oriented initial spin matrix because we are excluding the first number of MC cycles where having a different initial configuration would make a difference)

The findings presented here correspond with what we see in Fig.~\ref{fig:L20_ex_values}a-d where the curves are completely straight for a low temperature, giving the expected energy in almost every MC cycle, while the curves oscillate in the case of a high temperature, making the probability smaller.






\subsection{Looking for phase transitions}

\begin{figure*}[t!]
\begin{center}
\begin{tabular}{ccc}
  	\includegraphics[width=56mm,frame]{images/visualization_L80_T2_0_square.png}
	& \includegraphics[width=56mm,frame]{images/visualization_L80_T2_2_square.png}
	& \includegraphics[width=56mm,frame]{images/visualization_L80_T2_4_square.png} \\
	
	(a) $T = 2.0$					
	& (b) $T = 2.2$
	& (c) $T = 2.4$  \\[6pt]
\end{tabular}
\caption{Visualization of the spin matrix for a $80 \times 80$ lattice for different temperatures $T$. White = spin up, black = spin down.}\label{fig:visualization}
\end{center}
\end{figure*}

A phase transition in a magnetic system is defined as the transition between the ferromagnetic and paramagnetic phases of the system. The transition occurs at a critical point, often referred to as the Curie point, which is dependent on temperature. Beyond the critical point defined by a critical temperature $T_{\mathrm{C}}$ (see next subsection) the system will loose its magnetic properties and the magnetic spins will be randomly aligned in our case where there is no external magnetic field affecting the spin directions.

The spin matrix in the case of $L = 80$ for different temperatures is visualized in Fig.~\ref{fig:visualization}, where white squares represent upward spins and black squares represent downward spins. It shows clearly that the net magnetization $\langle {\cal M}\rangle$ is higher for low $T$ with most spins pointing in the same direction (in this case upwards), while the $\langle {\cal M}\rangle$ approaches zero at higher $T$ where the number of spins pointing upward and downward are about the same.

We see the same behaviour in Fig.~\ref{fig:ex_values_temp}b, which shows that $\langle {\cal M}\rangle$ decreases more and more for higher temperatures, with a particularly steep decrease around $T \approx 2.25$ and the value start to stabilize around $T \approx 2.35$ in the $L = 100$ case. Here $\langle {\cal M}\rangle$ approaches zero, meaning the net magnetization is expected to disappear more and more. This transition becomes steeper and steeper for increasing values of $L$.

These observations tells us that we might be witnessing a phase transition, going from an ordered, magnetic state to a random, nearly non-magnetized state. This transition is expected to be even clearer for higher values of $L$.

Around the same temperature range the susceptibility and heat capacity also exhibits a clear change in behaviour. In the case of the susceptibility in Fig.~\ref{fig:ex_values_temp}c we see that the peak sharpens more and more for higher values of $L$, pointing us towards a critical temperature of $T_{\mathrm{C}} \approx 2.3$. Similar behaviour can be seen for the heat capacity in Fig.~\ref{fig:ex_values_temp}d, though the peak is more ambiguous here, making the susceptibility a better measurement of the critical temperature. 

Both the susceptibility and the heat capacity are extensive quantities meaning they are proportional to the size of the system, $L$, which is clear from the graphs in Fig.~\ref{fig:ex_values_temp}c-d, where both quantities increase with the size of $L$.


\begin{figure*}[t]
\begin{center}
\begin{tabular}{c c c}
  	\includegraphics[width=85mm]{images/E_avg_temp.png}
	& \includegraphics[width=85mm]{images/M_avg_temp.png} \\
	(a) Energy					& (b) Magnetization  \\
	
	 \includegraphics[width=85mm]{images/chi_temp.png}
	& \includegraphics[width=85mm]{images/Cv_temp.png} \\
	(c) Susceptibility					& (d) Heat capacity  \\[6pt]
\end{tabular}
\caption{Expectation values per spin as a function of temperature.}\label{fig:ex_values_temp}
\end{center}
\end{figure*}





\subsection{Estimating the critical temperature $T_{\mathrm{C}}$}

Lastly, we want to estimate the critical temperature $T_{\mathrm{C}}$ in the thermodynamic limit $L \rightarrow \infty$ using Eq. (\ref{eq:T_C}). In our study of the behaviour of our model close to the critical temperature we look at the various expectation values as a function of $T$ for $L = 20, L = 40, L = 60$ and $L = 80$ for $T \in [2.0,2.4]$, shown in Fig.~\ref{fig:ex_values_temp}a-d. 

The calculated critical temperature $T_{\mathrm{C}}(L)$ obtained from the datasets by considering at which temperature the susceptibility in Fig.~\ref{fig:ex_values_temp}c peaks is listed in the following table for a step length of 0.05 in temperature:

\begin{center}
\begin{tabular}{ l l }
	\hline
	Lattice size $L$ 	& $T_{\mathrm{C}}(L)$\\
	\hline
	$10$				& 2.40	\\
	$20$				& 2.35	\\
	$40$				& 2.30 	\\
	$60$				& 2.30	\\
	$80$				& 2.30	\\
	$100$			& 2.30	\\
	\hline
\end{tabular}
\end{center}
Here we notice a problem: Our estimated value for $T_{\mathrm{C}}$ shows no sign of converging towards the exact value of 2.269 for larger values of $L$. Also, from Eq. (\ref{eg:T_C_estimate}) we see that because $T_{\mathrm{C}}(L)>T_{\mathrm{C}}$, we will never get closer to $T_{\mathrm{C}}$ no matter how large $L$ is, due to the step length. Assuming that $a$ is of the order unity, Eq. (\ref{eg:T_C_estimate}) would thus result in $T_{\mathrm{C}} = 2.3$ when $L\rightarrow\infty$ based on the results of our calculations. This is not a horrible estimate, in fact it has a relative error of only $1.36 \times 10^{-2}$ (if assuming $a = 1$). But we would like to decrease the temperature step length to see if we can do better.

We thus decreased the temperature step length to $0.01$ and ran our calculations again. This second time around we only considered temperatures in the range around the peak of the susceptibility in Fig.~\ref{fig:ex_values_temp}c, namely $T \in [2.25,2.35]$ and only for $L = 80$ and $L = 100$ due to timing issues. The new results was:

\begin{center}
\begin{tabular}{ l l l l }
	\hline
	Lattice size $L$ 	& $T_{\mathrm{C}}(L)$		& $T_{\mathrm{C}}(L \rightarrow \infty)$ 	& Rel. error\\
	\hline
	$80$				& 2.300					& 2.297							& $1.23 \times 10^{-2}$\\
	$100$			& 2.290					& 2.288							& $8.15 \times 10^{-3}$\\
	\hline
\end{tabular}
\end{center}
where we used the values from the two first columns to find that $a = 0.25$ by the method described in Sec.~\ref{subsec:T_C}, which lets us make an approximation for $T_{\mathrm{C}}(L \rightarrow \infty)$ by using Eq. (\ref{eg:T_C_estimate}), given in the third column. The relative error with respect to the exact value of $T_{\mathrm{C}} = 2.269$ is given in the fourth column. We see that the result for $L = 100$ is quite good, and that the result for $L = 80$ does not improve much from the case with the larger step length. 


\subsection{Parallelization}
As described previously, we parallelized our code using MPI. When using parallelization, we had to make sure to only allow \verb@rank = 0@ to print out the results of our calculations, to get readable files.

Here we have gathered some recordings of the time usage, using $n = 4$ processes on a 2,5 GHz Intel Core i5 processor (a four-year old MacBook Pro) for a temperature step length of 0.05:

\begin{center}
\begin{tabular}{ l r }
	\hline
	Lattice size $L$ 	& Time usage $t$\\
	\hline
	$10$				& 00m 57s		\\
	$20$				& 03m 52s		\\
	$40$				& 15m 23s 		\\
	$60$				& 48m 06s		\\
	$80$				& 2h 02m 00s	\\
	$100$			& 8h 46m 22s	\\
	\hline
\end{tabular}
\end{center}
These data is plotted in Fig.~\ref{fig:time_usage}, approximated by a $L^2$ curve which makes a good fit for lower values of $L$, and a $e^{L/9.5}$ curve which is a good approximation for higher values of $L$.

\begin{figure*}
\begin{center}
  	\includegraphics[width=90mm]{images/time_usage.png}
\caption{Time needed to run the program as a function of lattice size.}\label{fig:time_usage}
\end{center}
\end{figure*}

%\begin{center}
%	\includegraphics[width=90mm]{images/time_usage.png}	
%	\captionof{figure}{Time needed to run the program as a function of lattice size.}
%	\label{fig:time_usage}
%\end{center}
We also wanted to run our code on the super-computers in the Astrophysics department and compare the time usage with the time spent on the MacBook, but we had trouble making MPI work on these machines.




\section{Conclusions}

We have studied the phase transition of a spin system using the Ising model in two dimensions without an external magnetic field, running the calculations using the Monte Carlo method for different lattice sizes $L$. Our findings are described in the relevant parts of Sec.~\ref{sec:results}, but we briefly summarize here the quality of our findings and their dependence on the lattice size $L$ versus the time consumption of the code. 

From Fig.~\ref{fig:time_usage} it is evident that the time consumption of the code increases significantly with larger lattice size $L$, making it desirable to get away with as low a value for $L$ as possible. How low $L$ can we get away with?

We see that the lattice size $L$ is extremely important in some cases. Reviewing the results for the expectation values in Fig.~\ref{fig:ex_values_temp}a-d, the value of $\langle E \rangle$ does not vary much with lattice size, but the lattice size is highly important in the case of $\langle {\cal M}\rangle$, which can tell us whether a phase transition has occurred or not -- which is exactly what we want to study. It takes a value of $L = 60$ to create prominent peaks in the curves for the heat capacity and the susceptibility. 

The quality of our simulation can be discussed by comparing our numerical results to the exact value of the critical temperature $T_{\mathrm{C}}$. We found an estimate with a relative error of $1.36 \times 10^{-2}$ in the case of $L = 80$ with a temperature step length of 0.05, which is quite satisfactory, with an even better result in the case of $L = 100$ using a temperature step length of 0.01. However, the improvement is not very significant.

Summarizing these comments, we see that $L = 80$ would be a sufficient lattice size when trying to strike a balance between the numerical results and the time consumption of the computation. On a better computer the time consumption would not be that big of an issue, and a larger lattice size would be a better choice. We also see that a temperature step length of 0.05 gives decent results.





\section{List of codes}

The codes developed and used for this project are:\\

\noindent \verb@main.cpp@ -- main program (C++).

\noindent \verb@plotting.py@ -- plotting program which makes plots to study the total number of accepted configurations as a function of total number of MC cycles and as a function of temperature $T$ (Python).

\noindent \verb@visualization.py@ -- a provided program for running the Metropolis algorithm and plotting the spin matrix, as shown in Fig.~\ref{fig:visualization} (Python).

\end{multicols}

\end{document}
