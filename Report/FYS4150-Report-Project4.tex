\documentclass{article}

\usepackage[margin=0.5in,bottom=1in,footnotesep=1in]{geometry}

\usepackage{amsmath}


\usepackage{multicol}
\setlength{\columnsep}{1cm}
\usepackage[]{algorithm2e}

\usepackage{lipsum}% for dummy text
\usepackage[varg]{txfonts}
\usepackage{graphicx}
\usepackage{subcaption}
\usepackage{multirow}

\usepackage[font=small,labelfont={sf,bf}]{caption}

\usepackage{color}

\usepackage{titlesec}
\titleformat{\section}{\fontfamily{phv}\fontsize{12}{15}\bfseries}{\thesection}{1em}{}
\titleformat{\subsection}{\fontfamily{phv}\fontsize{10}{15}\itshape}{\thesubsection}{1em}{}
\titleformat{\subsubsection}{\fontfamily{phv}\fontsize{9}{15}\bfseries}{\thesubsubsection}{1em}{}


\title{\textbf{FYS4150 Project 4: \\The Ising model in two dimensions}}
\author{Marie Foss (\# 56), Maria Hammerstr{{\o}}m (\# 59)}
\date{}

\begin{document}

\maketitle

\begin{abstract}
	\noindent \lipsum[1]
	\vspace*{2ex}
	
	\noindent \textbf{Github:} \textit{https://github.com/mariahammerstrom/Project4}
	\vspace*{2ex}
\end{abstract}



\begin{multicols}{2}

\section{Introduction}

This project deals with the Ising model in two dimensions without an external magnetic field. The Ising model is a model to study phase transitions at finite temperature for magnetic systems. The model was invented in 1920\footnote{Lenz, W. (1920). "Beitr�ge zum Verst�ndnis der magnetischen Eigenschaften in festen K�rpern". \textit{Physikalische Zeitschrift} 21: 613-615.} and the analytical solution to the two-dimensional case was found in 1944\footnote{Onsager, Lars (1944). "Crystal statistics. I. A two-dimensional model with an order-disorder transition". \textit{Physical Review}, Series II 65 (3-4): 117-149.}.

\subsection{Mathematical foundation}
In the two-dimensional Ising model, the simplest form of the \textbf{energy} for a specific microstate $i$ is expressed as

\begin{equation}\label{eq:energy}
	E_i = -J \sum_{<kl>}^{N}s_k s_l,
\end{equation}
with  $s_k=\pm 1$ where $+ 1$ denotes spin up and $ - 1$ denotes spin down, $N$ is the total number of spins and $J$ is a coupling constant expressing the strength of the interaction between neighboring spins. The symbol $<kl>$ indicates that we sum over nearest neighbors only. We will assume that we have a ferromagnetic ordering, meaning $J> 0$. We will use periodic boundary conditions and the Metropolis algorithm only. 

We will be using the \textbf{Boltzmann probability distribution} (PDF) defined as

\begin{equation}
	P_i (\beta) = \frac{e^{-\beta E_i}}{Z},
\end{equation}
where $P_i$ expresses the probability of finding the system in a given microstate $i$, $\beta = 1/kT$ with $k$ the Boltzmann constant and $T$ the temperature, and with $Z$ as the \textbf{partition function} for the canonical ensemble defined as

\begin{equation}\label{eq:partition_func}
	Z = \sum_{i = 1}^{\infty} e^{-\beta E_i},
\end{equation}
summing over all micro states $i$. 

The \textbf{magnetic moment} of a given microstate is

\begin{equation}\label{eq:magnetization}
	M_i = \sum_j s_j.
\end{equation} 
Some quantities we are interested in calculating, are the \textbf{expection values} for the energy $<E>$ and magnetic moment $<M>$:

\begin{equation}
\begin{aligned}
	<E> &= \frac{1}{Z} \sum_i E_i e^{- \beta E_i} = kT^2 \frac{\partial \textrm{ ln } Z}{\partial T} = - \frac{\partial \textrm{ ln } Z}{\partial \beta}, \\
	<M> &= \frac{1}{Z} \sum_i M_i e^{- \beta E_i},
\end{aligned}
\end{equation}
as well as the variances for the energy $\sigma_E^2$ and for the magnetic moment $\sigma_M^2$:

\begin{equation}\label{eq:expect_values}
\begin{aligned}
	\sigma_E^2 &= <E^2> - <E>^2, \\
	\sigma_M^2 &= <M^2> - <M>^2.
\end{aligned}
\end{equation}
These values can be used to calculate the \textbf{heat capacity} of a fixed volume given by

\begin{equation}
	C_V = \frac{\sigma_E^2}{kT^2} = \frac{\partial <E>}{\partial T},
\end{equation}
and the \textbf{susceptibility}, which describes whether a material is attracted into or repelled out of a magnetic field, given by

\begin{equation}
	\chi = \frac{\sigma_M^2}{kT} = \frac{<M^2> - <M>^2}{kT}.
\end{equation}



\subsection{Analytical solution}
First we will assume that we only have two spins in each dimension, that is $L = 2$, where $L$ is the lattice length. The situation looks like this:

\begin{equation*}
\begin{aligned}
	\uparrow_{(1)} \quad \uparrow_{(2)}   \\
	\uparrow_{(3)} \quad \uparrow_{(4)}   
\end{aligned}
\end{equation*}
where an upward arrow denotes spin up. We will make use of \textbf{periodic boundary conditions}, which means that the neighbor to the right of a given spin $s_N$ takes the value of $s_1$. Similarly, the neighbor to the left of $s_1$ takes the value of $s_N$. This way of treating the boundaries are often used when approximating an infinite system that has a repeating structure. In our case, this mean we will treat our system as though it looked like this (where the original system is highlighted in magenta):

\begin{equation*}
\begin{aligned}
	\textcolor{black}{\uparrow_{(4)}} \quad \textcolor{black}{\uparrow_{(3)}} \quad \textcolor{black}{\uparrow_{(4)}} \quad \textcolor{black}{\uparrow_{(3)}} \\
	\textcolor{black}{\uparrow_{(2)}} \quad \textcolor{magenta}{\uparrow_{(1)}} \quad \textcolor{magenta}{\uparrow_{(2)}} \quad \textcolor{black}{\uparrow_{(1)}} \\
	\textcolor{black}{\uparrow_{(4)}} \quad \textcolor{magenta}{\uparrow_{(3)}} \quad \textcolor{magenta}{\uparrow_{(4)}} \quad \textcolor{black}{\uparrow_{(3)}} \\
	\textcolor{black}{\uparrow_{(2)}} \quad \textcolor{black}{\uparrow_{(1)}} \quad \textcolor{black}{\uparrow_{(2)}} \quad \textcolor{black}{\uparrow_{(1)}}
\end{aligned}
\end{equation*}
Closed form expression can be found for the partition function in Eq. (\ref{eq:partition_func}) and the corresponding expectation values for $E$, $|{\cal M}|$, the specific heat $C_V$ and the susceptibility $\chi$ as functions of $T$ using periodic boundary conditions. In the case $L = 2$ we can write Eq. (\ref{eq:energy}) for the different components as, making use of the periodic boundary conditions, giving

\begin{equation*}
\begin{aligned}
	E_1 &= - J (s_1 s_2 + s_1 s_3) = - 2J, \\
	E_2 &= - J (s_1 s_2 + s_2 s_4) = - 2J, \\
	E_3 &= - J (s_1 s_3 + s_3 s_4) = - 2J, \\
	E_4 &= - J (s_3 s_4 + s_2 s_4) = - 2J. 
\end{aligned}
\end{equation*}
Thus the total energy for the system is

\begin{equation*}
	E = E_1 + E_2 + E_3 + E_4 = - 8J.
\end{equation*}
If going through this exercise for different configurations of spin up and spin down, we find values for energies, degeneracies and magnetization for different configurations as shown in Table~\ref{table:quantities}, where magnetization is calculated from Eq. (\ref{eq:magnetization}).

\begin{table*}
\begin{center}
\begin{tabular}{ l r r r }\hline
	Number of spins up 			& Energy $E$	 			& Degeneracy $\Omega$		& Magnetization $M$		\\ \hline
	4 						& $- 8J$ 					& 1						& 4		 \\
	3 						& 0						& 4						& 2		 \\
	2						& 0						& 4						& 0		\\
	2						& $8J$					& 2						& 0		\\
	1						& 0						& 4						& -2		\\
	0						& $-8J$					& 1						& -4		\\
	\hline
\end{tabular}
\caption{Energy and magnetization for the two-dimensional Ising model with $N = 2 \times 2$ spins with periodic boundary conditions.}\label{table:quantities}
\end{center}
\end{table*}

These calculations can be used to find the closed-form expressions, starting with the partition function, which can be expressed as

\begin{equation}\label{eq:Z_analytic}
\begin{aligned}
	Z(\beta) 	&= \sum_E \Omega(E) e^{- \beta E} = 2 e^{8J\beta} + 2 e^{- 8J \beta} + 12 \\
			&= 4 \textrm{ cosh}(8J \beta) + 12.
\end{aligned}
\end{equation}
Then the expectation value can be written, using the expression in Eq. (\ref{eq:expect_values}), as

\begin{equation}\label{eq:E_avg_analytic}
	<E> = - \frac{8J \textrm{ sinh}(8J \beta)}{\textrm{cosh}(8J\beta) + 3} \approx - 8J \textrm{ tanh} (8J\beta),
\end{equation}
where we have used that cosh$(8J\beta) \gg 3$ in the approximation. Using this approximation gives

\begin{equation}\label{eq:C_V_analytic}
	C_V(\beta) = k \textrm{ } \bigg( \frac{8J\beta}{\textrm{cosh }(8J\beta)} \bigg)^2.
\end{equation}
Similarly, using the values for magnetization in Table~\ref{table:quantities} and Eq. (\ref{eq:magnetization}), gives

\begin{equation}\label{eq:M_avg_analytic}
\begin{aligned}
	<|M|> 	&= \frac{1}{Z} \bigg( 4e^{8J\beta} + 4\cdot2 + 4\cdot|-2| + |-4|e^{8J\beta} \bigg), \\
			&= \frac{8}{Z}(e^{8J\beta} + 2) \\
\end{aligned}
\end{equation}
and

\begin{equation}
\begin{aligned}
	<M> 		&= \frac{1}{Z} \bigg( 4e^{8J\beta} + 4\cdot2 - 4\cdot2 - 4e^{8J\beta} \bigg) = 0, \\
	%<M>^2 	&= 0 \\
	<M^2> 	&= \frac{1}{Z^2} \bigg( 4^2 e^{8J\beta} + 4\cdot2^2 + 4\cdot(-2)^2 + (-4)^2e^{8J\beta}   \bigg) \\
			&= \frac{32}{Z^2} \bigg( e^{8J\beta} + 1  \bigg).
\end{aligned}
\end{equation}
Thus

\begin{equation}\label{eq:chi_analytic}
	\chi(\beta) = \frac{<M^2> - <M>^2}{kT} = \frac{32 \beta}{Z^2} \bigg(e^{8J\beta} + 1  \bigg).
\end{equation}









\section{Methods}

We wrote a code for the Ising model which computes the mean energy $E$, mean magnetization $|{\cal M}|$, the specific heat $C_V$ and the susceptibility $\chi$ as functions of  $T$ using periodic boundary conditions for in the $x$ and $y$ directions. Using the Ising model in two dimensions, the number of configurations is given by $2^N$ with $N = L \times L$ number of spins for a lattice of length $L$. 

\subsection{Algorithm}
We will use the \textbf{Metropolis algorithm}. The algorithm goes as follows:

\begin{enumerate}
	\item Generate a random configuration in the lattice to create an initial state with energy $E_b$.
	\item Change the initial configuration by flipping for example one spin only. Compute the energy of this state $E_t$.
	\item Calculate $\Delta E = E_t - E_b$.
	\item \textit{The Metropolis test:} If $\Delta E \leq 0$ the new configuration is accepted, meaning the energy is lowered and we are moving towards the energy minimum at a given temperature. If $\Delta E > 0$, calculate $w = e^{- \beta \Delta E}$. If $w < r$ where $r$ is a random number, accept the new configuration. If else, keep the old configuration.
	\item Compute the new energy $E' = E_t 0 \Delta E$. Calculate various expectation values using this energy.
	\item Repeat steps $2-5$ for the chosen number of Monte Carlo cycles, meaning the number of times we should sweep through the lattice and summed over all spins.
	\item Compute the various expectation values by dividing by the total number of cycles (and possibly the number of spins).
\end{enumerate}


\subsection{Parallelization}
...



\section{Results}

Our numerical results from computing the mean energy $E$, mean magnetization $|{\cal M}|$, the specific heat $C_V$ and the susceptibility $\chi$ as functions of $T$ using periodic boundary conditions, can be compared with expressions given in Eq. (\ref{eq:Z_analytic}) - (\ref{eq:chi_analytic}) for a temperature $T=1.0$ (in units of $kT/J$). This gives the following results:

\begin{center}
\begin{tabular}{ l l l }\hline
	Quantity 								& Closed form	 				& Numerical		\\ \hline
	Mean energy $E$ 						& - 							& -		 \\
	Mean magnetization $|{\cal M}|$ 			& -							& -		 \\
	Specific heat $C_V$						& -							& - 		\\
	Susceptibility $\chi$						& -							& -		\\
	\hline
\end{tabular}
\end{center}
The number of Monte Carlo cycles needed to achieve good agreement is XXX.

Next we looked at the case of $L = 20$ spins in the $x$ and $y$ directions. In this case we wanted to perform a study of the time (or number of Monte Carlo cycles) we need before reaching an equilibrium situation and can start computing the various expectation values. 

A rough and study is done by plotting the various expectation values as a function of number of Monte Carlo cycles, for both $T = 1.0$ kT/J and T = $2.4$ kT/J, as shown in Fig. XXX. PLOT + COMMENTS!

We also study the total number of accepted configurations as a function of the total number of Monte Carlo cycles, shown in Fig. XXX. PLOT + COMMENTS! 

...

By simply counting the number of times a given energy appears in our computation we can compute the probability $w_i$. MORE. 




\section{Conclusions}

...





\section{List of codes}

The codes developed for this project are:\\

\noindent \verb@main.cpp@ -- main program (C++).

\noindent \verb@plotting.m@ -- plotting program which makes plots to study the total number of accepted configurations as a function of total number of Monte Carlo cycles and temperature $T$ (Python).

\end{multicols}

\end{document}
